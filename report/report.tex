
\documentclass[12pt]{article}
\usepackage[cm]{fullpage}
\usepackage[usenames,dvipsnames]{color}
\usepackage{sectsty}
\usepackage{listings}
\usepackage{color}
\usepackage{textpos}
\usepackage[usenames,dvipsnames,table]{xcolor}
\usepackage[small, bf]{caption}
\usepackage{amssymb,latexsym}
\usepackage{amsmath}
\usepackage{xltxtra,xunicode,xgreek}
\usepackage[colorlinks=true, linkcolor = blue, citecolor=blue]{hyperref}
\usepackage[ddmmyyyy]{datetime}
\usepackage{multirow}
\usepackage{tikz}
\usepackage{qtree}
\usepackage{float}
\usepackage{minted}
\usepackage[boxed]{algorithm2e}
\usepackage{textcomp}
\usepackage{fullpage} 
\usepackage{url} 
\usepackage{hyperref}
\usepackage{turnstile}

\definecolor{lightgray}{gray}{0.95}
\setmainfont[Mapping=tex-text]{CMU Concrete}
\setmonofont{Courier New}
\newcommand{\tab}{\hspace*{2em}}
\newcommand{\rom}[1]{\uppercase\expandafter{\romannumeral #1\relax}}
\setlength{\parindent}{0pt}
\renewcommand\listingscaption{Κώδικας}
\newcommand{\Llama}{\textit{Llama }}


\usemintedstyle{tango}
\SetKw{Kwdto}{downto}

\begin{document}
\begin{titlepage}
\begin{center}

\includegraphics[scale=0.3]{pyrforos.jpg}\\
ΕΘΝΙΚΟ ΜΕΤΣΟΒΙΟ ΠΟΛΥΤΕΧΝΕΙΟ \\
ΣΧΟΛΗ ΗΛΕΚΤΡΟΛΟΓΩΝ ΜΗΧΑΝΙΚΩΝ KΑΙ ΜΗΧΑΝΙΚΩΝ ΥΠΟΛΟΓΙΣΤΩΝ \\ 
\vspace{0.5em}

\medskip 

\def\doubleline{

    \vspace{0.1em}
    \line(1,0){530}\

    \vspace{-1.5em}
    \line(1,0){530}

}
\doubleline
\vspace{1.3em}

{\large \textbf{Μεταγλωτιστές}\\
 \medskip
8ο εξάμηνο, Ακαδημαϊκή περίοδος 2012 \\ \bigskip \medskip}

\vspace{1.5em}
{\LARGE \textbf{Υλοποίηση της γλώσσας \Llama\\}}
\vspace{1cm}
\includegraphics[scale=0.4]{llama.jpg}\\
\vfill
\begin{tabular}{l l}
Kωνσταντίνος Αθανασίου & 03108132\\
Νίκος Γιανναράκης & 03108054 \\
Ζωή Παρασκευοπούλου & 03108152 \\
\end{tabular}\\
\bigskip
\today
\end{center}
\end{titlepage}
\tableofcontents

\pagebreak

\section{Εισαγωγή}
H \Llama είναι μια γλώσσα η οποία συνδυάζει στοιχεία συναρτησιακού και προστακτικού προγραμματισμού και συντακτικά μοιάζει αρκετά με τη γλώσσα Caml. Η εργασία έχει υλοποιηθεί στη γλώσσα OCaml. Αναλυτικά οι προδιαγραφές τις γλώσσας βρίσκονται \href{http://courses.softlab.ntua.gr/compilers/2012a/llama2012.pdf}{εδώ}
\section{Υλοποίση}
Στην ενότητα αυτή θα παρουσιάσουμε συνοπτικά κάποια βασικά στοιχεία της υλοποίησης.
\subsection{Συντακτικός και Γραμματικός Αναλυτής}
Για τον συντακτικό αναλυτή χρησιμοποιήθηκε το εργαλείο \href{http://courses.softlab.ntua.gr/compilers/2012a/ocamlyacc-tutorial.pdf}{Ocamllex} και για τον γραμματικό     αναλυτή το εργαλείο \href{http://courses.softlab.ntua.gr/compilers/2012a/ocamlyacc-tutorial.pdf}{Ocamlyacc}.
\subsection{Σημασιολογική Ανάλυση}
Ο σημασιολογικός αναλυτής της \Llama αποτελείται κυρίως από την εξαγωγή τύπων καθώς και από τον έλεγχο κάποιων επιπλέον περιορισμών.
\subsubsection{Εξαγωγή Τύπων}
Η εξαγωγή τύπων βασίζεται στην παραγωγή περιορισμών, υπό τη μορφή εξισώσεων τύπων, κατά τη διάσχιση του ast, κάθε άγωστος τύπος αναπαριστάται από μια μοναδική μεταβλητή τύπου. Στο τέλος της διάσχισης καλείται η συνάρτηση unify η οποία λύνει τους περιορισμούς αναθέτει σε κάθε μεταβλητή τύπου έναν γνωστό τύπο. Σε περίπτωση όπου οι περιορισμοί δεν είναι επιλύσιμοι επιστρέφεται μήνυμα λάθους το οποίο επισημαίνει του δύο τύπους που δεν μπορούν να ενοποιηθούν.


Στην γλώσσα Llama οι τελεστές σύγκρισης υποστηρίζονται μόνο για τους τύπους int, float και char. Προκειμένου να ελέγξουμε στατικά αυτόν τον περιορισμό δημιουργούμε έναν καινούριο τύπο ord που αναπαριστά τους τύπους που υποστηρίζουν τελεστές διάταξης. Έτσι προκύπτει ο παρακάτω κανόνας τυποποίησης:

$$\frac{\Gamma  \vdash e_1 : ord \;\;\;\; \Gamma  \vdash e_2 : ord  }{\Gamma  \vdash e_1 \diamond e_2 : bool,\diamond \in \lbrace <,>, \leq, \geq\rbrace}$$


Μετά την δημιουργία των αντιστοίχων constraints η συνάρτηση unify συσσωρεύει τούς τύπους που προκύπτει από τους περιορισμούς ότι πρέπει να υποστηρίζουν διάταξη και αφού λύσει όλους τους περιορισμούς και εφαρμόσει όλες τις αντικαταστάσεις που προκύπτουν για τις μεταβλητές τύπων, ελέγχει αν οι συσσωρευμένοι τύποι είναι από τους 3 τύπους που υποστηρίζουν διάταξη, διαφορετικά γυρνάει μήνυμα λάθους. Με ανάλογο τρόπο ελέγχεται ότι ο τύπος επιστροφής των συναρτήσεων δεν είναι τύπος συνάρτησης κάτι που απαγορεύεται στη \Llama. 

%% TODO TODO TODΟ elegxos sunarthshs pou elegxei ta ord ston compiler einai lathos

\subsubsection{Έλεγχος επιπλέον περιορισμών}





\subsection{Πίνακες}
\subsection{Συναρτησεις Υψηλής Τάξης}
\subsection{Τύποι Οριζόμενοι από τον Προγραμματιστή}
\subsubsection{Αναπαράσταση}
\subsubsection{Υλοποίηση Δομικής Ισότητας}
\subsection{Control Flow Graph}
\subsubsection{Βελτιστοποιήσεις}
\end{document}
